\section{Evaluation}
\label{section:evaluation}

\todo{Organize/Revise the context; Add references of benchmarks; Revise tables; Discussions}

\begin{table*}[t]
\centering
\caption{Results of z3-Trau, cvc4, and z3 on string benchmarks}
\begin{tabular}{l | r r r | r r r | r r r | r r r}
\hline
\multirow{2}{*}{}   & \multicolumn{3}{|c|}{z3-Trau} & \multicolumn{3}{|c}{cvc4} & \multicolumn{3}{|c}{z3} & \multicolumn{3}{|c}{z3-str3} \\
			& sat & unsat & timeout/$\times$ & sat & unsat & timeout/$\times$ & sat & unsat & timeout/$\times$ & sat & unsat & timeout/$\times$ \\ \hline
PyEx		& 19586 & 3858 & 1969/8 & 19763 & 3834 & 0/1824 & 16581 & 3832 & 5008/0 & 3024 & 3839 & 16708/1850 \\ 
APLAS		&   128 &  287 &  185/0 & 205 &  221 & 174/0 &  13 &  100 & 486/1 & 38 & 111 & 93/358 \\ 
LeetCode	&   856 & 1784 &   26/0 & 860 & 1785 &  21/0 & 881 & 1785 & 0/0 & 670 & 1780 &  83/133 \\ 
StringFuzz	& 502 & 294 & 267/2 & 677 & 240 & 63/85 & 265 & 187 & 609/4 & 493 & 190 & 377/5 \\ 
cvc4\textsubscript{pred}	& 16 & 814 & 5/0 & 11 & 818 & 6/0 & 12 & 808 & 15/0 & 8 & 772 & 41/14 \\ 
cvc4\textsubscript{term}	& 13 & 1030 & 2/0 & 8 & 936 & 12/89 & 5 & 1021 & 19/0 & 16 & 958 & 53/18 \\ \hline
\end{tabular}
\label{table:base_benchmark}
\end{table*}


\begin{table*}[t]
\centering
\caption{Results of z3-Trau, cvc4, and z3 on str\_int benchmark}
\begin{tabular}{l | r r r | r r r | r r r | r r r}
\hline
\multirow{2}{*}{}   & \multicolumn{3}{|c|}{z3-Trau} & \multicolumn{3}{|c}{cvc4} & \multicolumn{3}{|c}{z3} & \multicolumn{3}{|c}{z3-str3} \\
			& sat & unsat & timeout/$\times$ & sat & unsat & timeout/$\times$ & sat & unsat & timeout/$\times$ & sat & unsat & timeout/$\times$ \\ \hline
full\_str\_int		& 3294 & 17088 & 1191/0 & 2185 & 16377 & 3011/0 & 422 & 16034 & 4131/986 & 2731 & 16832 & 2010/0 \\ 
filtered\_str\_int	& 3281 & 2912 & 1203/0 & 2210 & 2211 & 2975/0 & 2729 & 2655 & 2012/0 & 424 & 1944 & 4106/922 \\ \hline
\end{tabular}
\label{table:str_int_benchmark}
\end{table*}


We have implemented the proposed approach in a prototype tool \textsf{z3-Trau} as a part of the Z3 theorem prover. This approach allows us to efficiently solve not only formulae over string constraints, but also combine string constraints with constraints over other theories that Z3 supports. Furthermore, this approach allows us to more effectively handle the arithmetic constraints that are generated by the under-approximation module and eliminate the need to have our own parser for input formulae.

We compare \textsf{z3-Trau} with other state-of-the-art string solvers, namely CVC4~\cite{cvc4Tool}, Z3~\cite{z3}, \textsf{z3str3}~\cite{zheng2013z3}, Ostrich~\cite{chen2019decision} and \textsf{Trau+}~\cite{parosh2019chain}. We do not compare with Sloth \cite{sloth} since it does not support length constraints. We present data on two benchmark groups that demonstrate two points. First, only our tool can efficiently solve formulae that contain string-number and number-string conversion constraints and which can be a combination of constraints over multiple theories. Second, we show that our tool is competitive on well-known benchmarks that can be effectively handled by other existing solvers. The summary of experimenting for each benchmark group is given in Table~\ref{table:base_benchmark} and Table~\ref{table:str_int_benchmark}, respectively. All experiments were executed on a computer with 4-core CPU, 8 GiB RAM and Ubuntu 18.4-LVM OS. The timeout was set to 10s for each test. Columns with heading \texttt{sat}/\texttt{unsat} show numbers of solved formulae. Columns with heading \texttt{timeout} indicate the number of times the solver exceeded the time limit and columns with heading $\times$ show numbers of instances for which the particular solver either reported unknown or reported incorrect answer or crashed.

Occasionally, during the experiments, there was a case where a solver reported a different answer to a test than the other solvers. To decide which solver reported the incorrect answer, we have developed our own validator. Our validator takes the model from the solver who reported that the test is satisfiable and assigns the corresponding values according to the returned model to the variables in the test. The modified test is re-evaluated by solvers. If the solvers decide that the modified test is satisfiable, the answer of the solvers that initially reported unsat is evaluated as incorrect.

\paragraph{A Comparison with other tools on conversion-less benchmarks.}
This group of benchmarks consists of PyEx, APLAS, LeetCode, StringFuzz, cvc4\textsubscript{pred} and cvc4\textsubscript{term}, which are benchmarks that were obtained by using an existing tool or generated by other groups. In this group of benchmarks, we would like to show that the performance of our tools is not only comparable to the performance of other tools, but in some cases even better.

The first benchmark is called PyEx~\cite{pyex} according to the same-named tool, which is a symbolic executor designed for Python developers to achieve high-coverage testing. This benchmark was obtained from the CVC4 group who ran PyEx on a test suite from 4 popular Python packages: httplib2, pip, pymongo, and requests. PyEx benchmark consists of 25421 tests which contain formulae with diverse string constraints.

The second benchmark is called APLAS that was created by authors of \textsf{$Kepler_{22}$}~\cite{aplas}. The benchmark includes a total of 600 hand-drafted tests (298 satisfiable and 302 unsatisfiable) involving looping word equations (Both sides of the string equality have a common variable) and length constraints over strings. 

The next benchmark is called LeetCode~\cite{LeetCode} that was obtained by extracting constraints from Python's testing solutions provided by LeetCode platform. They provide many programming examples and their solutions gathered from technical interviews for companies. Leetcode consists of 881 satisfiable and 1785 unsatisfiable tests that, like PyEx, contain diverse string constraints.

StringFuzz is our next bechmark that is named after a fuzzer~\cite{StringFuzz} for automatically generating SMT-LIB string constraints. We used StringFuzz to generate 1065 tests including word (dis)equalities, regular membership and arithmetic constraints. 

The last two benchmarks, called $\text{cvc4}_{\text{pred}}$ and $\text{cvc4}_{\text{term}}$, are obtained from cvc4 group~\cite{termEQ}. This set of benchmarks consists of the verification of term equivalences over strings and includes various string constraints including string-number and number-string conversion constraints.





\hide{
In this section, we compare our implementation z3-Trau with other SMT tools cvc4, z3, and z3-str3 as evauation. To show the general performance of z3-Trau, we compare z3-Trau with other string solvers on selected string benchmarks: PyEx is a benchmark obtained from symbolic execution of Python code[]; APLAS is a benchmark involving looping word equations[]; 
LeetCode is obtiained from concolic testing LeetCode solutions written in Python code; StringFuzz is a benchmark of instance SMT-lib string problems generated by StringFuzz generator tool[]; cvc4\textsubscript{pred} and cvc4\textsubscript{term} are benchmarks provided by the cvc4 development team[]. Table~\ref{table:base_benchmark} shows the result of the comparison. The experiments are conducted with machines of the following specifications: 4-core CPU, 8GB RAM, Ubuntu 18.4-LVM OS. We set the timeout is to 10 seconds. Because the amount of problems is very large, we ran these experiments separately on several machines with the same specification on a computer cluster. The results are either sat, unsat, timeout, or $\times$. In case $\times$, the result may be unknown, error, or exception.
}

\textbf{Comparison according to Table 1......}


To evaluate our strategy for string-number/number-string conversion, we also prepared a benchmark \texttt{str\_int}\footnote{\url{https://github.com/plfm-iis/str_int_benchmarks}}. It is collected from two sources of Python programs that use ttexttt{int()} function: Leetcode solutions written in Python and Python core libraries. We concolic tested these Python programs by \texttt{Py-Conbyte}\footnote{\url{https://github.com/spencerwuwu/py-conbyte}}, our concolic tester for Python programs. The SMT queries during the concolic testing are collected as our benchmark. To be more precise in evalutation, we have two versions of \texttt{str\_int} benchmark: \texttt{full\_str\_int} and \texttt{filtered\_str\_int}. \texttt{full\_str\_int} is the original benchmark we collected (i.e. from Python programs using \texttt{int()}); \texttt{filtered\_str\_int} is a subset of \texttt{fill\_str\_int}. We filtered out problems that cvc4 says unsat while the unsat cores do not contain \texttt{str.to.int} or \texttt{int.to.str}. The results of experiment on \texttt{str\_int} benchmark is listed in Table~\ref{table:str_int_benchmark}. The experiments are conducted under the same condition as the experiments on other benchmarks.


\textbf{Comparison according to Table 2.....}



\hide{
\begin{table}[]
\caption{Results of z3-Trau, cvc4, and z3 on full\_str\_int benchmark}
\begin{tabular}{|r|r|r|r|r|r|r|}
\hline
Tool		& sat & unsat & u.k. & t.o. & err. & misc \\ \hline\hline
z3-Trau		& 3289 & 17089 & 0 & 1195 & 0 & 0 \\ 
cvc4		& 2185 & 16377 & 0 & 3011 & 0 & 0 \\ 
z3seq		& 2716 & 16831 & 0 & 2026 & 0 & 0 \\ 
z3str3		& 422 & 16034 & 634 & 4131 & 347 & 5 \\ \hline
\end{tabular}
\label{table:full_str_int}
\end{table}

Table~\ref{table:filtered_str_int} shows the comparison on filtered\_str\_int.  The total amount of cases in filtered\_str\_int is 7396.

\begin{table}[]
\caption{Results of z3-Trau, cvc4, and z3 on filtered\_str\_int benchmark}
\begin{tabular}{|r|r|r|r|r|r|r|}
\hline
Tool		& sat & unsat & u.k. & t.o. & err. & misc \\ \hline\hline
z3-Trau		& 3281 & 2912 & 0 & 1203 & 0 & 0 \\ 
cvc4		& 2210 & 2211 & 0 & 2975 & 0 & 0 \\ 
z3seq		& 2729 & 2655 & 0 & 2012 & 0 & 0 \\ 
z3str3		& 424 & 1944 & 587 & 4106 & 330 & 5 \\ \hline
\end{tabular}
\label{table:filtered_str_int}
\end{table}
}